% For some odd reason, LaTeX warns that the global option for font size
% is unused. Need to debug this
% Wednesday 07 March 2018 02:01:02 AM IST
\documentclass{letter}

% Using the geometry package to set the margin, orientation, and the
% paper dimensions. Update according to you requirements
\usepackage[a4paper, portrait, margin=1in]{geometry}

% Using the fontspec package to set the document main font.
% I tend to use a system-wide available sans-serif font. Whichever
% sails your boat, works here.
% After a handful of experiments, I stuck to using Gentium Book Basic.
% The font sizes used in the custom commands reflect the sizes that
% seem good for me at 100% on an Ubuntu box at 1920x1080, with evince
% Play around with your specifics to arrive at your personal choice.
\usepackage{fontspec}
\setmainfont[Ligatures=TeX]{Gentium Book Basic}

% \usepackage[utf8]{inputenc}
\usepackage[english]{babel}

% Without the hyperref package and the hidelinks directive, LaTeX adds
% rather ugly borders around the links in the document.
\usepackage[hidelinks]{hyperref}
\hypersetup{pdfproducer=XeLaTeX,%
            pdfauthor={YourName},%
            pdftitle={YourTitle},%
            pdfsubject={YourSubject},%
            pdfkeywords={YourKeywords},
            pdfcreator=XeLaTeX
}

% Set the line height at 1.5
\renewcommand{\baselinestretch}{1.5}

% Add custom commands for the document

\newcommand{\smallTxt}[1] {
  \begin{small}#1\end{small}
}

\newcommand{\normalTxt}[1] {
  \begin{normalsize}#1\end{normalsize}
}

\newcommand{\largeTxt}[1] {
  \begin{large}#1\end{large}
}

% Sort of, minor components?
\newcommand{\titledLink}[2] {
    \href{#1}{#2}
}

% Component to add an email address with a subject added. Ensures proper
% thought is put into the communication and there is a context in mind.
\newcommand{\emailWithSubject}[2] {
    \titledLink{
        mailto:#2?subject=#1
    } {
        \sized{12pt}{#2}
    }
}

% Your headline - a short note on your expertise/role
\newcommand{\headline}[1] {
    \begin{center}\sized{13pt}{#1}\\[40pt]\end{center}
}

% Convenience wrapper since I tend to at times forget \textbf
\newcommand{\bold}[1] {
    \textbf{#1}
}

% sized takes in two arguments - a font size and the text to render
% Renders the text taken in, in the requested font size.
\newcommand{\sized}[2] {
    \fontsize{#1}{#1}\selectfont{#2}
}

% Component for your work experience title/organisation.
\newcommand{\experienceRole}[1] {
    \sized{13.5pt}{\bold{\noindent#1}}
}

% Component for your work experience duration/period.
\newcommand{\experienceDuration}[1] {
    % Use hfill to push the duration text to the right end of the line
    \hfill\sized{12.5pt}{#1}\\[10pt]
}

\newcommand{\experience}[3] {
    % Use a flushleft to left align the role and duration in a single
    % line. This is a bit buggy. The noindent somehow does not work here
    % and a minor indentation seems to be introduced. Needs a fix
    % Wednesday 07 March 2018 02:06:13 AM IST

    % The component itself renders thus
    % <Title/Position/Organisation>        <Duration/Period>
    % <Description/role/work done while in the position>
    \begin{flushleft}
        \experienceRole{#1}\experienceDuration{#2}
    \end{flushleft}
    #3
}

\newcommand{\experienceLine}[1] {
    % Individual paragraphs within each experience to use this directive
    % Sets a 12.5pt size for the lines. Any further aestheic changes
    % to be added according to environment.
    \fontsize{12pt}{12pt}\selectfont
    #1
}

\newcommand{\userLink}[2] {
    \titledLink{#1}{\sized{14pt}{#2}}
}

\newcommand{\userLinkSmaller}[2] {
    \titledLink{#1}{\sized{12pt}{#2}}
}

\newcommand{\schools}[1] {
    \vspace{15pt}#1
}

\newcommand{\school}[2] {
    \begin{flushright}
        \textbf{\sized{13pt}{#1}}\\#2
    \end{flushright}
}


\begin{document}

% Title/links block - A large name block, oneline breadcrumb links block, email block, and a final phone number block.

\begin{center}\textbf{\sized{18pt}{Resume Owner}}\\[7.5pt]

% Keep web links stuck to a single line in a breadcrumb format
% Homepage | LinkedIn | Portfolio/Dribble/Github sounds good here.
% TODO: Generate a usable component out of this structure
\userLink{https://home}{Home} |
\userLink{https://linkedin}{LinkedIn} |
\userLink{https://github}{Github}
\\[2.5pt]

% I've come to use a subject hard-coded email here. For one, it makes sure
% there is a particular focus on the context I send this resume over for.
% Extending on, is a good thing from the receiver's shoes, the effort.
\emailWithSubject {Re: Communication subject} {user@email.com}

% Phone number block - Stick to a single phone number that you are available at, not multiple ones here.
\sized{12pt}{Phone}
\end{center}

% Draws a line of 1pt thickness spanning the width of the page text
\noindent\rule{\textwidth}{1pt}

% Opening line - Keep crisp and encompassing all experience.
\headline{
  Headline - Introduce yourself here
}

\experience{
  Role / Organisation
} {
  Role duration
} {
    \experienceLine {
        Something you did while being at this organisation/the role.
    }

    \experienceLine {
        Some more changes you brought in, while building things here.
    }

}

\experience{
  Role / Organisation
} {
  Role duration
} {
    \experienceLine {
        Something you did while being at this organisation/the role.
    }

    \experienceLine {
        Things you learnt while picking things up for work here.
    }

}

\schools {
    \school {
        Program you went to
    } {
        School name - Graduation date/duration
    }

    \school {
        Any other program you went to
    } {
        School name - Graduation date/duration
    }
}

\end{document}